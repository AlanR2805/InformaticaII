\documentclass[8pt, a4paper]{article} %en clase va 8pt, 11pt, 12pt, a4paper, letterpaper
  \usepackage[left=12mm, top=0.5in, bottom=5in]{geometry}  
  \author{Alan Rueda} %Autor
  \date{1 de Febrero de 2018} %Fecha
  
\begin{document} %Comenzar el doc
  \title{Tarea no. 2} % nombre del titulo
   \maketitle % reafirmar titulo
  
    \section{Respuesta de la pregunta de confusion} %seccion 2
    \begin{sloppypar} %se usa cuando hay palabras muy largas y no queden espacios en blanco
    Este extraño resultado se debe a que la tarea "deberes1" esta asignado para las dos propiedades o funciones, logicamente cuando una se complete la otra se completara, porque ambos puntean a la misma tarea.\\
    \\
    Observacion: Ademas de ese error, se puede apreciar un error de sintaxis en Persona alonzo = new Persona("Alonzo". "Church"), en vez del punto se debe de colocar una coma.\\
    \end{sloppypar}  
    
        
\end{document}